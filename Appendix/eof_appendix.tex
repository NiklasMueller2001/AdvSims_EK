%!TEX root = ../Lab_report.tex
\onecolumn
\lstset{basicstyle=\footnotesize\ttfamily, breaklines=true}
\section{Appendix}
\subsection{Electro-Osmotic Flow simulation parameters}
\begin{table}[H]
	\caption{System parameter in SI units and simulation units}
	\centering
	\begin{tabular}{l c c c c c}
		\toprule
		parameter  & value (SI units) & value (simulation units)\\
		\midrule
		channel width $d$&\SI{12}{nm}&$12\,$[x]\\
		counter ion charge $q$&\SI{1}{e}&$1\,$[q]\\
		thermal energy $k_{\text{B}}T$&$k_{\text{B}}\SI{300}{K}\,$&1[E]\\
		vacuum permittivity $\varepsilon_\text{0}$&\SI{8.85e-10}{s^2kgm^3}&\SI{1.428e-3}{}$\,$[q]$^2$[t]$^2$/[m][x]$^3$\\
		relative solvent permittivity5  $\varepsilon_\text{r}$&\SI{78.54}{}&78.54\\
		external electrical field $E$&\SI{0.646e9}{Vm^{-1}}&$25\,$[E]/[q][x]\\
		solvent density $\rho$&\SI{1000}{kg\per m^3}&$26.18\,$[m]/[x]$^3$\\
		kinematic solvent viscosity $\nu$&\SI{8.23e-8}{m^2\per s}&$0.25\,$[x]$^2$/[t]\\
		counterion frictional coefficient $\gamma$&\SI{1.910}{kg\per s}&$15\,$[m]/[t]\\
		\bottomrule
	\end{tabular}	
	\label{tab:params}
\end{table}
\subsection{Electro-Osmotic FLow script}
In the following, the script for the simulation of the electro-osmotic flow is given.
First, the required modules are imported and the constants are defined.  


\begin{lstlisting}[language=python]
	
import espressomd
import espressomd.lb
import espressomd.electrostatics
import espressomd.observables
import espressomd.accumulators
import espressomd.polymer
import espressomd.visualization
import espressomd.shapes
import espressomd.io
import espressomd.checkpointing
import numpy as np
import logging
import threading
import time
import tqdm
import os
	
	
BOX_L = [16,16,23]
SLIT_WIDTH = 12
L_BJERRUM = 0.7095 # 7.095 Angstrom
DELTA_T = 0.01
SEED = 42
KT=1
EPSILON_0=(1.428e-3)
EPSILON_R=78.54
E_FIELD=25
RHO_A = 26.18
VISCOCITY = 0.25
GAMMA = 15
#constraints
CONST1 = 1.5
CONST2 = -14.5
#Lattize Boltzmann walls
LBB1 = 1.5
LBB2 = -14.5
#Wall particle positions
WALLP1 = 1
WALLP2 = 15
E_FORCE = np.array([E_FIELD,0,0])
np.random.seed(SEED)


\end{lstlisting}

In the next step, the functions that build the system are defined. The counter ions are placed between the walls by calling \texttt{generate\_counterions} and the overlap  is removed by the function \texttt{steepest\_descent\_counterions}. Afterwards the wall particles are placed and the and the constraints are set.
\begin{lstlisting}[language=python]

def build_system():
	logging.info("Build System ...")
	
	system = espressomd.System(box_l=BOX_L)
	system.periodicity = [True, True, True]
	system.cell_system.skin = 0.4
	system.time_step = DELTA_T
	system.non_bonded_inter[0, 1].wca.set_params(epsilon=1., sigma=1)
	system.non_bonded_inter[0, 0].wca.set_params(epsilon=1, sigma=1.)
	system.non_bonded_inter[0, 2].wca.set_params(epsilon=1., sigma=1)
	return system

def generate_counterions(system, n_counter_ions, seed):
	logging.info("Creating Counterions ...")
	for _ in range(n_counter_ions):
	counter_ions_pos = np.array([np.random.rand()* BOX_L[0]
	,np.random.rand()* BOX_L[0],
	9*np.random.rand()+3.5])
	system.part.add(pos=counter_ions_pos, q=1., type=0)


def steepest_descent_counterions(system,steps):
	logging.info("Removing overlap Counterions ...")
	logging.info("BEFORE MINIMIZATION " +str(system.analysis.energy()["total"]))
	system.integrator.set_steepest_descent(
	f_max=0,
	gamma=30.,
	max_displacement=0.01)
	system.integrator.run(steps)
	system.integrator.set_vv()
	logging.info("AFTER MINIMIZATION "+str(system.analysis.energy()["total"]))


def generate_wall_particle(system,walld, wallu,p_per_site,n_counter_ions):
	logging.info("Creating Wall Particles ...")
	
	xs = np.linspace(0,16,p_per_site+1)
	xs += (xs[1]-xs[0])/2
	xss, yss = np.meshgrid(xs[:-1],xs[:-1])
	for xs,ys in zip(xss,yss):
	for x,y in zip(xs,ys):
	system.part.add(pos=[x,y,walld], q=-(n_counter_ions/2)/(p_per_site**2), type=1,fix =[True,True,True])
	system.part.add(pos=[x,y,wallu], q=-(n_counter_ions/2)/(p_per_site**2), type=1, fix =[True,True,True])

def generate_constraint(const1,const2):
	bottom_constraint = espressomd.shapes.Wall(dist=const1, normal=[0, 0, 1])
	top_constraint = espressomd.shapes.Wall(dist=const2, normal=[0, 0, -1])
	
	system.constraints.add(shape=bottom_constraint, particle_type=2,penetrable=False)
	system.constraints.add(shape=top_constraint, particle_type=2,penetrable=False)
	

\end{lstlisting}
Then, the electrostatic and the hydrodynamic interactions are set. Additionally, there is the functions \texttt{turn\_on\_e\_filed} that defines the force that acts on the counter ions.
\begin{lstlisting}[language=python]


def set_electrostatics(system, l_bjerrum, seed):#4.2#
	logging.info("Turn on electrostatics")
	p3m = espressomd.electrostatics.P3M(
	prefactor=l_bjerrum,accuracy =1e-4,verbose = False)
	
	elc = espressomd.electrostatics.ELC(actor=p3m,
	gap_size=5,
	maxPWerror=1e-4,
	)
	system.actors.add(elc)


def set_hydrodynamic_interaction(system,wall_l, wall_r,fd,com_flag, seed):#4.2
	system.galilei.kill_particle_motion()
	system.galilei.galilei_transform()
	bottom_wall = espressomd.shapes.Wall(dist=wall_l, normal=[0, 0, 1])
	top_wall = espressomd.shapes.Wall(dist=wall_r, normal=[0, 0, -1])
	bottom_boundary = espressomd.lbboundaries.LBBoundary(shape=bottom_wall)
	top_boundary = espressomd.lbboundaries.LBBoundary(shape=top_wall)
	system.lbboundaries.add(bottom_boundary)
	system.lbboundaries.add(top_boundary)
	logging.info("Set hydrodynamic interactions 4.2 CPU")
	lbfunc = espressomd.lb.LBFluid
	
	
	lbf = lbfunc(agrid=1,
	dens=26.18,visc=0.25,
	tau=system.time_step,
	kT=1,seed=seed)
	
	system.actors.add(lbf)
	system.thermostat.set_lb(LB_fluid=lbf, seed=seed, gamma=15)
return lbf



def turn_on_e_field(system,force):
	system.part.select(type=0).ext_force = force
#generate an accumulator for a density observable	
def calculate_dens_accu(system,obs):
	accu =espressomd.accumulators.TimeSeries(obs=obs, delta_N=100)
	system.auto_update_accumulators.add(accu)
return accu


\end{lstlisting}
Finally the functions are called in the previously given order and the system is equilibrated. Afterwards, the accumulators for the flow and the density profile are defined and the production run is performed
\begin{lstlisting}[language=python]
if __name__ == "__main__":
system = build_system()
generate_counterions(system,128,SEED)
generate_wall_particle(system,WALLP1,WALLP2,8,128)
generate_constraint(CONST1,CONST2)
steepest_descent_counterions(system,1000)
set_electrostatics(system,L_BJERRUM,SEED)
turn_on_e_field(system,1*E_FORCE)
lbf = set_hydrodynamic_interaction(system, LBB1, LBB2,0,SEED)
visualizer= espressomd.visualization.openGLLive(system)
eq_steps = int(5e4)
logging.info('Equilibrating')
system.integrator.run(eq_steps)

accumulator1 = calculate_dens_accu(system, espressomd.observables.LBVelocityProfile(
n_x_bins=1,
n_y_bins=1,
n_z_bins=26,
min_x=0,
min_y=0,
min_z=0,
max_x=16,
max_y=16,
max_z=16, 
allow_empty_bins=True,
sampling_offset_x=0,
sampling_offset_y=0,
sampling_offset_z=0,
sampling_delta_x=16,
sampling_delta_y=16,
sampling_delta_z=1
))

accumulator2 = calculate_dens_accu(system, espressomd.observables.FluxDensityProfile(
n_x_bins=1,
n_y_bins=1,
n_z_bins=32,
min_x=0,
min_y=0,
min_z=0,
max_x=16,
max_y=16,
max_z=16,
allow_empty_bins=True,
sampling_offset_x=0,
sampling_offset_y=0,
sampling_offset_z=0,
sampling_delta_x=1,
sampling_delta_y=1,
sampling_delta_z=0.5,
ids = system.part.select(type=0).id
))

logging.info('Prod run started')
total_stepps = int(1e5)
system.integrator.run(total_stepps)
\end{lstlisting}

\begin{lstlisting}[language=python]
	
\end{lstlisting}