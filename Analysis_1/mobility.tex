%!TEX root = ../Lab_report.tex

The observed mobilities are in disagreement with the expected behavior. The monotonic increase of $\mu$ with the number of monomers is only expected for short polymers. In general, in the steady state, the electric force on the polymer $\vec{F}_E = Q_\text{eff} \vec{E}$ is balanced by the counteracting solvent drag force $\vec{F}_D = -\Gamma_\text{eff} \vec{v}$. The experimentally observed constant mobility for long chains can be explained with the free draining-picture. The polymer is assumed to be penetrated by counterions, which drag along surrounding solvent. This results in the destruction of long range hydrodynamical interactions. The effective friction on the polymer scales linearly with the number of monomers for long chains, i.e.
\begin{equation}
	\Gamma_\text{eff} = \Gamma N.
\end{equation}
Together with the effective charge predicted by Manning theory of
\begin{equation}
	Q_\text{eff} = N / \xi
\end{equation}
a constant mobility
\begin{equation}
	\mu = \frac{v}{E} = \frac{Q_\text{eff}}{\Gamma_\text{eff}} = \frac{N}{\Gamma N \xi} = \frac{1}{\Gamma \xi}
\end{equation}
is obtained. In the transition region between short and long chains a mobility maximum is expected. Our results indicate an inaccurate description of hydrodynamic interaction in the system, which could be the reason for the unexpected behavior of the mobility.